\documentclass{scrartcl}

% current annoyance: this will be fixed
% by the next update of agda.fmt
\def\textmu{}

%% ODER: format ==         = "\mathrel{==}"
%% ODER: format /=         = "\neq "
%
%
\makeatletter
\@ifundefined{lhs2tex.lhs2tex.sty.read}%
  {\@namedef{lhs2tex.lhs2tex.sty.read}{}%
   \newcommand\SkipToFmtEnd{}%
   \newcommand\EndFmtInput{}%
   \long\def\SkipToFmtEnd#1\EndFmtInput{}%
  }\SkipToFmtEnd

\newcommand\ReadOnlyOnce[1]{\@ifundefined{#1}{\@namedef{#1}{}}\SkipToFmtEnd}
\usepackage{amstext}
\usepackage{amssymb}
\usepackage{stmaryrd}
\DeclareFontFamily{OT1}{cmtex}{}
\DeclareFontShape{OT1}{cmtex}{m}{n}
  {<5><6><7><8>cmtex8
   <9>cmtex9
   <10><10.95><12><14.4><17.28><20.74><24.88>cmtex10}{}
\DeclareFontShape{OT1}{cmtex}{m}{it}
  {<-> ssub * cmtt/m/it}{}
\newcommand{\texfamily}{\fontfamily{cmtex}\selectfont}
\DeclareFontShape{OT1}{cmtt}{bx}{n}
  {<5><6><7><8>cmtt8
   <9>cmbtt9
   <10><10.95><12><14.4><17.28><20.74><24.88>cmbtt10}{}
\DeclareFontShape{OT1}{cmtex}{bx}{n}
  {<-> ssub * cmtt/bx/n}{}
\newcommand{\tex}[1]{\text{\texfamily#1}}	% NEU

\newcommand{\Sp}{\hskip.33334em\relax}


\newcommand{\Conid}[1]{\mathit{#1}}
\newcommand{\Varid}[1]{\mathit{#1}}
\newcommand{\anonymous}{\kern0.06em \vbox{\hrule\@width.5em}}
\newcommand{\plus}{\mathbin{+\!\!\!+}}
\newcommand{\bind}{\mathbin{>\!\!\!>\mkern-6.7mu=}}
\newcommand{\rbind}{\mathbin{=\mkern-6.7mu<\!\!\!<}}% suggested by Neil Mitchell
\newcommand{\sequ}{\mathbin{>\!\!\!>}}
\renewcommand{\leq}{\leqslant}
\renewcommand{\geq}{\geqslant}
\usepackage{polytable}

%mathindent has to be defined
\@ifundefined{mathindent}%
  {\newdimen\mathindent\mathindent\leftmargini}%
  {}%

\def\resethooks{%
  \global\let\SaveRestoreHook\empty
  \global\let\ColumnHook\empty}
\newcommand*{\savecolumns}[1][default]%
  {\g@addto@macro\SaveRestoreHook{\savecolumns[#1]}}
\newcommand*{\restorecolumns}[1][default]%
  {\g@addto@macro\SaveRestoreHook{\restorecolumns[#1]}}
\newcommand*{\aligncolumn}[2]%
  {\g@addto@macro\ColumnHook{\column{#1}{#2}}}

\resethooks

\newcommand{\onelinecommentchars}{\quad-{}- }
\newcommand{\commentbeginchars}{\enskip\{-}
\newcommand{\commentendchars}{-\}\enskip}

\newcommand{\visiblecomments}{%
  \let\onelinecomment=\onelinecommentchars
  \let\commentbegin=\commentbeginchars
  \let\commentend=\commentendchars}

\newcommand{\invisiblecomments}{%
  \let\onelinecomment=\empty
  \let\commentbegin=\empty
  \let\commentend=\empty}

\visiblecomments

\newlength{\blanklineskip}
\setlength{\blanklineskip}{0.66084ex}

\newcommand{\hsindent}[1]{\quad}% default is fixed indentation
\let\hspre\empty
\let\hspost\empty
\newcommand{\NB}{\textbf{NB}}
\newcommand{\Todo}[1]{$\langle$\textbf{To do:}~#1$\rangle$}

\EndFmtInput
\makeatother
%
%
%
%
%
%
% This package provides two environments suitable to take the place
% of hscode, called "plainhscode" and "arrayhscode". 
%
% The plain environment surrounds each code block by vertical space,
% and it uses \abovedisplayskip and \belowdisplayskip to get spacing
% similar to formulas. Note that if these dimensions are changed,
% the spacing around displayed math formulas changes as well.
% All code is indented using \leftskip.
%
% Changed 19.08.2004 to reflect changes in colorcode. Should work with
% CodeGroup.sty.
%
\ReadOnlyOnce{polycode.fmt}%
\makeatletter

\newcommand{\hsnewpar}[1]%
  {{\parskip=0pt\parindent=0pt\par\vskip #1\noindent}}

% can be used, for instance, to redefine the code size, by setting the
% command to \small or something alike
\newcommand{\hscodestyle}{}

% The command \sethscode can be used to switch the code formatting
% behaviour by mapping the hscode environment in the subst directive
% to a new LaTeX environment.

\newcommand{\sethscode}[1]%
  {\expandafter\let\expandafter\hscode\csname #1\endcsname
   \expandafter\let\expandafter\endhscode\csname end#1\endcsname}

% "compatibility" mode restores the non-polycode.fmt layout.

\newenvironment{compathscode}%
  {\par\noindent
   \advance\leftskip\mathindent
   \hscodestyle
   \let\\=\@normalcr
   \let\hspre\(\let\hspost\)%
   \pboxed}%
  {\endpboxed\)%
   \par\noindent
   \ignorespacesafterend}

\newcommand{\compaths}{\sethscode{compathscode}}

% "plain" mode is the proposed default.
% It should now work with \centering.
% This required some changes. The old version
% is still available for reference as oldplainhscode.

\newenvironment{plainhscode}%
  {\hsnewpar\abovedisplayskip
   \advance\leftskip\mathindent
   \hscodestyle
   \let\hspre\(\let\hspost\)%
   \pboxed}%
  {\endpboxed%
   \hsnewpar\belowdisplayskip
   \ignorespacesafterend}

\newenvironment{oldplainhscode}%
  {\hsnewpar\abovedisplayskip
   \advance\leftskip\mathindent
   \hscodestyle
   \let\\=\@normalcr
   \(\pboxed}%
  {\endpboxed\)%
   \hsnewpar\belowdisplayskip
   \ignorespacesafterend}

% Here, we make plainhscode the default environment.

\newcommand{\plainhs}{\sethscode{plainhscode}}
\newcommand{\oldplainhs}{\sethscode{oldplainhscode}}
\plainhs

% The arrayhscode is like plain, but makes use of polytable's
% parray environment which disallows page breaks in code blocks.

\newenvironment{arrayhscode}%
  {\hsnewpar\abovedisplayskip
   \advance\leftskip\mathindent
   \hscodestyle
   \let\\=\@normalcr
   \(\parray}%
  {\endparray\)%
   \hsnewpar\belowdisplayskip
   \ignorespacesafterend}

\newcommand{\arrayhs}{\sethscode{arrayhscode}}

% The mathhscode environment also makes use of polytable's parray 
% environment. It is supposed to be used only inside math mode 
% (I used it to typeset the type rules in my thesis).

\newenvironment{mathhscode}%
  {\parray}{\endparray}

\newcommand{\mathhs}{\sethscode{mathhscode}}

% texths is similar to mathhs, but works in text mode.

\newenvironment{texthscode}%
  {\(\parray}{\endparray\)}

\newcommand{\texths}{\sethscode{texthscode}}

% The framed environment places code in a framed box.

\def\codeframewidth{\arrayrulewidth}
\RequirePackage{calc}

\newenvironment{framedhscode}%
  {\parskip=\abovedisplayskip\par\noindent
   \hscodestyle
   \arrayrulewidth=\codeframewidth
   \tabular{@{}|p{\linewidth-2\arraycolsep-2\arrayrulewidth-2pt}|@{}}%
   \hline\framedhslinecorrect\\{-1.5ex}%
   \let\endoflinesave=\\
   \let\\=\@normalcr
   \(\pboxed}%
  {\endpboxed\)%
   \framedhslinecorrect\endoflinesave{.5ex}\hline
   \endtabular
   \parskip=\belowdisplayskip\par\noindent
   \ignorespacesafterend}

\newcommand{\framedhslinecorrect}[2]%
  {#1[#2]}

\newcommand{\framedhs}{\sethscode{framedhscode}}

% The inlinehscode environment is an experimental environment
% that can be used to typeset displayed code inline.

\newenvironment{inlinehscode}%
  {\(\def\column##1##2{}%
   \let\>\undefined\let\<\undefined\let\\\undefined
   \newcommand\>[1][]{}\newcommand\<[1][]{}\newcommand\\[1][]{}%
   \def\fromto##1##2##3{##3}%
   \def\nextline{}}{\) }%

\newcommand{\inlinehs}{\sethscode{inlinehscode}}

% The joincode environment is a separate environment that
% can be used to surround and thereby connect multiple code
% blocks.

\newenvironment{joincode}%
  {\let\orighscode=\hscode
   \let\origendhscode=\endhscode
   \def\endhscode{\def\hscode{\endgroup\def\@currenvir{hscode}\\}\begingroup}
   %\let\SaveRestoreHook=\empty
   %\let\ColumnHook=\empty
   %\let\resethooks=\empty
   \orighscode\def\hscode{\endgroup\def\@currenvir{hscode}}}%
  {\origendhscode
   \global\let\hscode=\orighscode
   \global\let\endhscode=\origendhscode}%

\makeatother
\EndFmtInput
%
%
\ReadOnlyOnce{agda.fmt}%


\RequirePackage[T1]{fontenc}
\RequirePackage[utf8x]{inputenc}
\RequirePackage{ucs}
\RequirePackage{amsfonts}

\providecommand\mathbbm{\mathbb}

% TODO: Define more of these ...
\DeclareUnicodeCharacter{737}{\textsuperscript{l}}
\DeclareUnicodeCharacter{8718}{\ensuremath{\blacksquare}}
\DeclareUnicodeCharacter{8759}{::}
\DeclareUnicodeCharacter{9669}{\ensuremath{\triangleleft}}
\DeclareUnicodeCharacter{8799}{\ensuremath{\stackrel{\scriptscriptstyle ?}{=}}}
\DeclareUnicodeCharacter{10214}{\ensuremath{\llbracket}}
\DeclareUnicodeCharacter{10215}{\ensuremath{\rrbracket}}

% TODO: This is in general not a good idea.
\providecommand\textepsilon{$\epsilon$}
\providecommand\textmu{$\mu$}


%Actually, varsyms should not occur in Agda output.

% TODO: Make this configurable. IMHO, italics doesn't work well
% for Agda code.

\renewcommand\Varid[1]{\mathord{\textsf{#1}}}
\let\Conid\Varid
\newcommand\Keyword[1]{\textsf{\textbf{#1}}}
\EndFmtInput



\usepackage[T1]{fontenc}
% \usepackage[utf8]{inputenc} 
\usepackage[polish]{babel} 
\usepackage{listings}

\newtheorem{zadanie}{Zadanie}

\author{Wojciech Jedynak \and Paweł Wieczorek}
\title{Ćwiczenia z Agdy - Lista 1.}

\begin{document}
\lstset{language=Haskell}

\maketitle

\section{Uwaga}

Lista zadań jest dostępna w dwu wariantach: .pdf i .lagda. Wersję .lagda
można otworzyć w edytorze tekstu i uzupełniać brakujące fragmenty bez przepisywania
wszystkiego.

\begin{hscode}\SaveRestoreHook
\column{B}{@{}>{\hspre}l<{\hspost}@{}}%
\column{E}{@{}>{\hspre}l<{\hspost}@{}}%
\>[B]{}\Keyword{module}\;\Conid{Exercises}\;\Keyword{where}{}\<[E]%
\ColumnHook
\end{hscode}\resethooks

\section{Podstawy Izomorfizmu Curry'ego-Howarda}

Fałsz zdefiniowaliśmy w Agdzie jako typ pusty:

\begin{hscode}\SaveRestoreHook
\column{B}{@{}>{\hspre}l<{\hspost}@{}}%
\column{E}{@{}>{\hspre}l<{\hspost}@{}}%
\>[B]{}\Keyword{data}\;\Varid{⊥}\;\mathbin{:}\;\Conid{Set}\;\Keyword{where}{}\<[E]%
\\[\blanklineskip]%
\>[B]{}\Varid{⊥-elim}\;\mathbin{:}\;\{\mskip1.5mu \Conid{A}\;\mathbin{:}\;\Conid{Set}\mskip1.5mu\}\;\Varid{→}\;\Varid{⊥}\;\Varid{→}\;\Conid{A}{}\<[E]%
\\
\>[B]{}\Varid{⊥-elim}\;(){}\<[E]%
\ColumnHook
\end{hscode}\resethooks

Możemy teraz wyrazić negację w standardowy sposób: jako funkcję w zbiór pusty.

\begin{hscode}\SaveRestoreHook
\column{B}{@{}>{\hspre}l<{\hspost}@{}}%
\column{E}{@{}>{\hspre}l<{\hspost}@{}}%
\>[B]{}\Varid{¬\char95 }\;\mathbin{:}\;\Conid{Set}\;\Varid{→}\;\Conid{Set}{}\<[E]%
\\
\>[B]{}\Varid{¬}\;\Conid{A}\;\mathrel{=}\;\Conid{A}\;\Varid{→}\;\Varid{⊥}{}\<[E]%
\ColumnHook
\end{hscode}\resethooks

\begin{zadanie}
Udowodnij, że p ⇒ ¬¬p, czyli dokończ poniższą definicję:

\begin{hscode}\SaveRestoreHook
\column{B}{@{}>{\hspre}l<{\hspost}@{}}%
\column{E}{@{}>{\hspre}l<{\hspost}@{}}%
\>[B]{}\Varid{pnnp}\;\mathbin{:}\;\{\mskip1.5mu \Conid{A}\;\mathbin{:}\;\Conid{Set}\mskip1.5mu\}\;\Varid{→}\;\Conid{A}\;\Varid{→}\;\Varid{¬}\;\Varid{¬}\;\Conid{A}{}\<[E]%
\\
\>[B]{}\Varid{pnnp}\;\mathrel{=}\;\{\mskip1.5mu \mathbin{!!}\mskip1.5mu\}{}\<[E]%
\ColumnHook
\end{hscode}\resethooks

Czy potrafisz udowodnić implikację w drugą stronę?

\end{zadanie}

\begin{zadanie}

Udowodnij prawo kontrapozycji, czyli pokaż że (A ⇒ B) ⇒ (¬ B ⇒ ¬ A). Czy potrafisz udowodnić twierdzenie odwrotne?

\end{zadanie}

\begin{zadanie}

Zdefiniuj typ, który będzie odpowiadał formule atomowej ⊤. Jakiemu typowi z języków programowania on odpowiada?

\end{zadanie}

\begin{zadanie}

Polimorficzne pary, czyli odpowiednik koniukcji możemy zdefiniować następująco:

\begin{hscode}\SaveRestoreHook
\column{B}{@{}>{\hspre}l<{\hspost}@{}}%
\column{3}{@{}>{\hspre}l<{\hspost}@{}}%
\column{E}{@{}>{\hspre}l<{\hspost}@{}}%
\>[B]{}\Keyword{data}\;\Varid{\char95 ∧\char95 }\;(\Conid{A}\;\Conid{B}\;\mathbin{:}\;\Conid{Set})\;\mathbin{:}\;\Conid{Set}\;\Keyword{where}{}\<[E]%
\\
\>[B]{}\hsindent{3}{}\<[3]%
\>[3]{}\Varid{pair}\;\mathbin{:}\;(\Varid{a}\;\mathbin{:}\;\Conid{A})\;\Varid{→}\;(\Varid{b}\;\mathbin{:}\;\Conid{B})\;\Varid{→}\;\Conid{A}\;\Varid{∧}\;\Conid{B}{}\<[E]%
\ColumnHook
\end{hscode}\resethooks

Udowodnij reguły eliminacji oraz prawo przemienności tj. zdefiniuj funkcje fst, snd i swap:

\begin{hscode}\SaveRestoreHook
\column{B}{@{}>{\hspre}l<{\hspost}@{}}%
\column{E}{@{}>{\hspre}l<{\hspost}@{}}%
\>[B]{}\Varid{fst}\;\mathbin{:}\;\{\mskip1.5mu \Conid{A}\;\Conid{B}\;\mathbin{:}\;\Conid{Set}\mskip1.5mu\}\;\Varid{→}\;\Conid{A}\;\Varid{∧}\;\Conid{B}\;\Varid{→}\;\Conid{A}{}\<[E]%
\\
\>[B]{}\Varid{fst}\;\mathrel{=}\;\{\mskip1.5mu \mathbin{!!}\mskip1.5mu\}{}\<[E]%
\\[\blanklineskip]%
\>[B]{}\Varid{snd}\;\mathbin{:}\;\{\mskip1.5mu \Conid{A}\;\Conid{B}\;\mathbin{:}\;\Conid{Set}\mskip1.5mu\}\;\Varid{→}\;\Conid{A}\;\Varid{∧}\;\Conid{B}\;\Varid{→}\;\Conid{B}{}\<[E]%
\\
\>[B]{}\Varid{snd}\;\mathrel{=}\;\{\mskip1.5mu \mathbin{!!}\mskip1.5mu\}{}\<[E]%
\\[\blanklineskip]%
\>[B]{}\Varid{swap}\;\mathbin{:}\;\{\mskip1.5mu \Conid{A}\;\Conid{B}\;\mathbin{:}\;\Conid{Set}\mskip1.5mu\}\;\Varid{→}\;\Conid{A}\;\Varid{∧}\;\Conid{B}\;\Varid{→}\;\Conid{B}\;\Varid{∧}\;\Conid{A}{}\<[E]%
\\
\>[B]{}\Varid{swap}\;\mathrel{=}\;\{\mskip1.5mu \mathbin{!!}\mskip1.5mu\}{}\<[E]%
\ColumnHook
\end{hscode}\resethooks

\end{zadanie}

\begin{zadanie}

Zdefiniuj typ sumy rozłącznej (czyli Haskellowy typ Either). Jakiemu spójnikowi logicznemu
odpowiada ten typ? Udowodnij przemienność tego spójnika.

\end{zadanie}

\begin{zadanie}

Korzystając z typów z poprzednich dwóch zadań, sformułuj i spróbuj udowodnić prawa deMorgana znane
z logiki klasycznej. Które z nich zachodzą w logice kostruktywnej?

\end{zadanie}


\section{Typ identycznościowy, czyli równość w języku}

Aby wydawać i dowodzić sądy o równości różnych bytów, zdefiniowaliśmy tzw. typ identycznościowy.
Typ ustalonych a,b ∈ A, a ≡ b jest zamieszkały wtw, gdy a i b obliczają się do tego samego
elementu kanonicznego (wartości).

\begin{hscode}\SaveRestoreHook
\column{B}{@{}>{\hspre}l<{\hspost}@{}}%
\column{3}{@{}>{\hspre}l<{\hspost}@{}}%
\column{E}{@{}>{\hspre}l<{\hspost}@{}}%
\>[B]{}\Keyword{infix}\;\Varid{5}\;\Varid{\char95 ≡\char95 }{}\<[E]%
\\[\blanklineskip]%
\>[B]{}\Keyword{data}\;\Varid{\char95 ≡\char95 }\;\{\mskip1.5mu \Conid{A}\;\mathbin{:}\;\Conid{Set}\mskip1.5mu\}\;\mathbin{:}\;\Conid{A}\;\Varid{→}\;\Conid{A}\;\Varid{→}\;\Conid{Set}\;\Keyword{where}{}\<[E]%
\\
\>[B]{}\hsindent{3}{}\<[3]%
\>[3]{}\Varid{refl}\;\mathbin{:}\;\{\mskip1.5mu \Varid{a}\;\mathbin{:}\;\Conid{A}\mskip1.5mu\}\;\Varid{→}\;\Varid{a}\;\Varid{≡}\;\Varid{a}{}\<[E]%
\ColumnHook
\end{hscode}\resethooks

Różność elementów definiujemy jako... zaprzeczenie równości:

\begin{hscode}\SaveRestoreHook
\column{B}{@{}>{\hspre}l<{\hspost}@{}}%
\column{E}{@{}>{\hspre}l<{\hspost}@{}}%
\>[B]{}\Varid{\char95 ≢\char95 }\;\mathbin{:}\;\{\mskip1.5mu \Conid{A}\;\mathbin{:}\;\Conid{Set}\mskip1.5mu\}\;\Varid{→}\;\Conid{A}\;\Varid{→}\;\Conid{A}\;\Varid{→}\;\Conid{Set}{}\<[E]%
\\
\>[B]{}\Varid{a}\;\Varid{≢}\;\Varid{b}\;\mathrel{=}\;\Varid{¬}\;(\Varid{a}\;\Varid{≡}\;\Varid{b}){}\<[E]%
\ColumnHook
\end{hscode}\resethooks

\begin{zadanie}

Pokazaliśmy już jak udowodnić niektóre właśności równości:

\begin{hscode}\SaveRestoreHook
\column{B}{@{}>{\hspre}l<{\hspost}@{}}%
\column{E}{@{}>{\hspre}l<{\hspost}@{}}%
\>[B]{}\Varid{symm}\;\mathbin{:}\;\{\mskip1.5mu \Conid{A}\;\mathbin{:}\;\Conid{Set}\mskip1.5mu\}\;\Varid{→}\;(\Varid{a}\;\Varid{b}\;\mathbin{:}\;\Conid{A})\;\Varid{→}\;\Varid{a}\;\Varid{≡}\;\Varid{b}\;\Varid{→}\;\Varid{b}\;\Varid{≡}\;\Varid{a}{}\<[E]%
\\
\>[B]{}\Varid{symm}\;\Varid{a}\;\Varid{.a}\;\Varid{refl}\;\mathrel{=}\;\Varid{refl}{}\<[E]%
\\[\blanklineskip]%
\>[B]{}\Varid{subst}\;\mathbin{:}\;\{\mskip1.5mu \Conid{A}\;\mathbin{:}\;\Conid{Set}\mskip1.5mu\}\;\Varid{→}\;(\Conid{P}\;\mathbin{:}\;\Conid{A}\;\Varid{→}\;\Conid{Set})\;\Varid{→}\;(\Varid{a}\;\Varid{b}\;\mathbin{:}\;\Conid{A})\;\Varid{→}\;\Varid{a}\;\Varid{≡}\;\Varid{b}\;\Varid{→}\;\Conid{P}\;\Varid{a}\;\Varid{→}\;\Conid{P}\;\Varid{b}{}\<[E]%
\\
\>[B]{}\Varid{subst}\;\Conid{P}\;\Varid{a}\;\Varid{.a}\;\Varid{refl}\;\Conid{Pa}\;\mathrel{=}\;\Conid{Pa}{}\<[E]%
\ColumnHook
\end{hscode}\resethooks

Udowodnij dwie dodatkowe (bardzo przydatne) właśności:

\begin{hscode}\SaveRestoreHook
\column{B}{@{}>{\hspre}l<{\hspost}@{}}%
\column{E}{@{}>{\hspre}l<{\hspost}@{}}%
\>[B]{}\Varid{trans}\;\mathbin{:}\;\{\mskip1.5mu \Conid{A}\;\mathbin{:}\;\Conid{Set}\mskip1.5mu\}\;\Varid{→}\;(\Varid{a}\;\Varid{b}\;\Varid{c}\;\mathbin{:}\;\Conid{A})\;\Varid{→}\;\Varid{a}\;\Varid{≡}\;\Varid{b}\;\Varid{→}\;\Varid{b}\;\Varid{≡}\;\Varid{c}\;\Varid{→}\;\Varid{a}\;\Varid{≡}\;\Varid{c}{}\<[E]%
\\
\>[B]{}\Varid{trans}\;\mathrel{=}\;\{\mskip1.5mu \mathbin{!!}\mskip1.5mu\}{}\<[E]%
\\[\blanklineskip]%
\>[B]{}\Varid{cong}\;\mathbin{:}\;\{\mskip1.5mu \Conid{A}\;\Conid{B}\;\mathbin{:}\;\Conid{Set}\mskip1.5mu\}\;\Varid{→}\;(\Conid{P}\;\mathbin{:}\;\Conid{A}\;\Varid{→}\;\Conid{B})\;\Varid{→}\;(\Varid{a}\;\Varid{b}\;\mathbin{:}\;\Conid{A})\;\Varid{→}\;\Varid{a}\;\Varid{≡}\;\Varid{b}\;\Varid{→}\;\Conid{P}\;\Varid{a}\;\Varid{≡}\;\Conid{P}\;\Varid{b}{}\<[E]%
\\
\>[B]{}\Varid{cong}\;\mathrel{=}\;\{\mskip1.5mu \mathbin{!!}\mskip1.5mu\}{}\<[E]%
\ColumnHook
\end{hscode}\resethooks


\end{zadanie}


\section{Wartości boolowskie}

Wartości boolowskie zdefiniowaliśmy następująco:

\begin{hscode}\SaveRestoreHook
\column{B}{@{}>{\hspre}l<{\hspost}@{}}%
\column{3}{@{}>{\hspre}l<{\hspost}@{}}%
\column{9}{@{}>{\hspre}l<{\hspost}@{}}%
\column{E}{@{}>{\hspre}l<{\hspost}@{}}%
\>[B]{}\Keyword{data}\;\Conid{Bool}\;\mathbin{:}\;\Conid{Set}\;\Keyword{where}{}\<[E]%
\\
\>[B]{}\hsindent{3}{}\<[3]%
\>[3]{}\Varid{false}\;\mathbin{:}\;\Conid{Bool}{}\<[E]%
\\
\>[B]{}\hsindent{3}{}\<[3]%
\>[3]{}\Varid{true}\;{}\<[9]%
\>[9]{}\mathbin{:}\;\Conid{Bool}{}\<[E]%
\ColumnHook
\end{hscode}\resethooks

\begin{zadanie}

Uzupełnij poniższą definicję tak, aby otrzymać funkcję negacji boolowskiej.

\begin{hscode}\SaveRestoreHook
\column{B}{@{}>{\hspre}l<{\hspost}@{}}%
\column{11}{@{}>{\hspre}l<{\hspost}@{}}%
\column{E}{@{}>{\hspre}l<{\hspost}@{}}%
\>[B]{}not\;\mathbin{:}\;\Conid{Bool}\;\Varid{→}\;\Conid{Bool}{}\<[E]%
\\
\>[B]{}not\;\Varid{false}\;\mathrel{=}\;\{\mskip1.5mu \mathbin{!!}\mskip1.5mu\}{}\<[E]%
\\
\>[B]{}not\;\Varid{true}\;{}\<[11]%
\>[11]{}\mathrel{=}\;\{\mskip1.5mu \mathbin{!!}\mskip1.5mu\}{}\<[E]%
\ColumnHook
\end{hscode}\resethooks

Udowodnij, że Twoja definicja spełnia następujące własności:

\begin{hscode}\SaveRestoreHook
\column{B}{@{}>{\hspre}l<{\hspost}@{}}%
\column{E}{@{}>{\hspre}l<{\hspost}@{}}%
\>[B]{}\Varid{not-has-no-fixed-points}\;\mathbin{:}\;(\Varid{b}\;\mathbin{:}\;\Conid{Bool})\;\Varid{→}\;not\;\Varid{b}\;\Varid{≢}\;\Varid{b}{}\<[E]%
\\
\>[B]{}\Varid{not-has-no-fixed-points}\;\mathrel{=}\;\{\mskip1.5mu \mathbin{!!}\mskip1.5mu\}{}\<[E]%
\\[\blanklineskip]%
\>[B]{}\Varid{not-is-involutive}\;\mathbin{:}\;(\Varid{b}\;\mathbin{:}\;\Conid{Bool})\;\Varid{→}\;not\;(not\;\Varid{b})\;\Varid{≡}\;\Varid{b}{}\<[E]%
\\
\>[B]{}\Varid{not-is-involutive}\;\mathrel{=}\;\{\mskip1.5mu \mathbin{!!}\mskip1.5mu\}{}\<[E]%
\ColumnHook
\end{hscode}\resethooks



\end{zadanie}

\section{Liczby naturalne}

Na wykładzie zdefiniowaliśmy liczby naturalne z dodawaniem następująco:

\begin{hscode}\SaveRestoreHook
\column{B}{@{}>{\hspre}l<{\hspost}@{}}%
\column{3}{@{}>{\hspre}l<{\hspost}@{}}%
\column{7}{@{}>{\hspre}l<{\hspost}@{}}%
\column{8}{@{}>{\hspre}l<{\hspost}@{}}%
\column{E}{@{}>{\hspre}l<{\hspost}@{}}%
\>[B]{}\Keyword{data}\;\Conid{ℕ}\;\mathbin{:}\;\Conid{Set}\;\Keyword{where}{}\<[E]%
\\
\>[B]{}\hsindent{3}{}\<[3]%
\>[3]{}\Varid{zero}\;\mathbin{:}\;\Conid{ℕ}{}\<[E]%
\\
\>[B]{}\hsindent{3}{}\<[3]%
\>[3]{}\Varid{suc}\;{}\<[8]%
\>[8]{}\mathbin{:}\;\Conid{ℕ}\;\Varid{→}\;\Conid{ℕ}{}\<[E]%
\\[\blanklineskip]%
\>[B]{}\mbox{\enskip\{-\# BUILTIN NATURAL ℕ  \#-\}\enskip}{}\<[E]%
\\
\>[B]{}\mbox{\enskip\{-\# BUILTIN ZERO zero  \#-\}\enskip}{}\<[E]%
\\
\>[B]{}\mbox{\enskip\{-\# BUILTIN SUC suc  \#-\}\enskip}{}\<[E]%
\\[\blanklineskip]%
\>[B]{}\Keyword{infix}\;\Varid{6}\;\Varid{\char95 +\char95 }{}\<[E]%
\\[\blanklineskip]%
\>[B]{}\Varid{\char95 +\char95 }\;\mathbin{:}\;\Conid{ℕ}\;\Varid{→}\;\Conid{ℕ}\;\Varid{→}\;\Conid{ℕ}{}\<[E]%
\\
\>[B]{}\Varid{zero}\;{}\<[7]%
\>[7]{}\Varid{+}\;\Varid{m}\;\mathrel{=}\;\Varid{m}{}\<[E]%
\\
\>[B]{}\Varid{suc}\;\Varid{n}\;\Varid{+}\;\Varid{m}\;\mathrel{=}\;\Varid{suc}\;(\Varid{n}\;\Varid{+}\;\Varid{m}){}\<[E]%
\ColumnHook
\end{hscode}\resethooks

\begin{zadanie}
Pamiętając, że wg ICH indukcja = rekursja, udowodnij następujące własności dodawania:

\begin{hscode}\SaveRestoreHook
\column{B}{@{}>{\hspre}l<{\hspost}@{}}%
\column{E}{@{}>{\hspre}l<{\hspost}@{}}%
\>[B]{}\Varid{plus-right-zero}\;\mathbin{:}\;(\Varid{n}\;\mathbin{:}\;\Conid{ℕ})\;\Varid{→}\;\Varid{n}\;\Varid{+}\;\Varid{0}\;\Varid{≡}\;\Varid{n}{}\<[E]%
\\
\>[B]{}\Varid{plus-right-zero}\;\mathrel{=}\;\{\mskip1.5mu \mathbin{!!}\mskip1.5mu\}{}\<[E]%
\\[\blanklineskip]%
\>[B]{}\Varid{plus-suc-n-m}\;\mathbin{:}\;(\Varid{n}\;\Varid{m}\;\mathbin{:}\;\Conid{ℕ})\;\Varid{→}\;\Varid{suc}\;(\Varid{n}\;\Varid{+}\;\Varid{m})\;\Varid{≡}\;\Varid{n}\;\Varid{+}\;\Varid{suc}\;\Varid{m}{}\<[E]%
\\
\>[B]{}\Varid{plus-suc-n-m}\;\mathrel{=}\;\{\mskip1.5mu \mathbin{!!}\mskip1.5mu\}{}\<[E]%
\ColumnHook
\end{hscode}\resethooks

\end{zadanie}

\begin{zadanie}
Korzystając z poprzedniego zadania, udowodnij przemienność dodawania:

\begin{hscode}\SaveRestoreHook
\column{B}{@{}>{\hspre}l<{\hspost}@{}}%
\column{E}{@{}>{\hspre}l<{\hspost}@{}}%
\>[B]{}\Varid{plus-commutative}\;\mathbin{:}\;(\Varid{n}\;\Varid{m}\;\mathbin{:}\;\Conid{ℕ})\;\Varid{→}\;\Varid{n}\;\Varid{+}\;\Varid{m}\;\Varid{≡}\;\Varid{m}\;\Varid{+}\;\Varid{n}{}\<[E]%
\\
\>[B]{}\Varid{plus-commutative}\;\mathrel{=}\;\{\mskip1.5mu \mathbin{!!}\mskip1.5mu\}{}\<[E]%
\ColumnHook
\end{hscode}\resethooks

\end{zadanie}


\begin{zadanie}
Zdefiniuj mnożenie i potęgowanie dla liczb naturalnych.

\begin{hscode}\SaveRestoreHook
\column{B}{@{}>{\hspre}l<{\hspost}@{}}%
\column{22}{@{}>{\hspre}l<{\hspost}@{}}%
\column{E}{@{}>{\hspre}l<{\hspost}@{}}%
\>[B]{}\Keyword{infix}\;\Varid{7}\;\Varid{\char95 *\char95 }{}\<[E]%
\\
\>[B]{}\Keyword{infix}\;\Varid{8}\;\Varid{\char95 \char94 \char95 }{}\<[E]%
\\[\blanklineskip]%
\>[B]{}\Varid{\char95 *\char95 }\;\mathbin{:}\;\Conid{ℕ}\;\Varid{→}\;\Conid{ℕ}\;\Varid{→}\;\Conid{ℕ}{}\<[E]%
\\
\>[B]{}\Varid{n}\;\Varid{*}\;\Varid{m}\;\mathrel{=}\;\{\mskip1.5mu \mathbin{!!}\mskip1.5mu\}{}\<[E]%
\\[\blanklineskip]%
\>[B]{}\Varid{\char95 \char94 \char95 }\;\mathbin{:}\;\Conid{ℕ}\;\Varid{→}\;\Conid{ℕ}\;\Varid{→}\;\Conid{ℕ}{}\<[E]%
\\
\>[B]{}\Varid{n}\;\mathbin{\uparrow}\;\Varid{m}\;\mathrel{=}\;\{\mskip1.5mu \mathbin{!!}\mskip1.5mu\}{}\<[22]%
\>[22]{}\mbox{\onelinecomment  tutaj też ma być daszek :-)}{}\<[E]%
\ColumnHook
\end{hscode}\resethooks

Jeśli masz ochotę, to udowodnij przemienność mnożenia.

\end{zadanie}

\begin{zadanie}

Udowodnij, że zero ≢ suc (zero).

\end{zadanie}

\begin{zadanie}

Udowodnij następującą własność ("prawo skracania"):

\begin{hscode}\SaveRestoreHook
\column{B}{@{}>{\hspre}l<{\hspost}@{}}%
\column{E}{@{}>{\hspre}l<{\hspost}@{}}%
\>[B]{}\Varid{strip-suc}\;\mathbin{:}\;(\Varid{n}\;\Varid{m}\;\mathbin{:}\;\Conid{ℕ})\;\Varid{→}\;\Varid{suc}\;\Varid{n}\;\Varid{≡}\;\Varid{suc}\;\Varid{m}\;\Varid{→}\;\Varid{n}\;\Varid{≡}\;\Varid{m}{}\<[E]%
\\
\>[B]{}\Varid{strip-suc}\;\mathrel{=}\;\{\mskip1.5mu \mathbin{!!}\mskip1.5mu\}{}\<[E]%
\ColumnHook
\end{hscode}\resethooks


\end{zadanie}



\section{Wektory}

Przypomnijmy definicję wektorów:

\begin{hscode}\SaveRestoreHook
\column{B}{@{}>{\hspre}l<{\hspost}@{}}%
\column{3}{@{}>{\hspre}l<{\hspost}@{}}%
\column{7}{@{}>{\hspre}l<{\hspost}@{}}%
\column{E}{@{}>{\hspre}l<{\hspost}@{}}%
\>[B]{}\Keyword{data}\;\Conid{Vec}\;(\Conid{A}\;\mathbin{:}\;\Conid{Set})\;\mathbin{:}\;\Conid{ℕ}\;\Varid{→}\;\Conid{Set}\;\Keyword{where}{}\<[E]%
\\
\>[B]{}\hsindent{3}{}\<[3]%
\>[3]{}[\mskip1.5mu \mskip1.5mu]\;{}\<[7]%
\>[7]{}\mathbin{:}\;\Conid{Vec}\;\Conid{A}\;\Varid{0}{}\<[E]%
\\
\>[B]{}\hsindent{3}{}\<[3]%
\>[3]{}\Varid{\char95 ∷\char95 }\;\mathbin{:}\;\{\mskip1.5mu \Varid{n}\;\mathbin{:}\;\Conid{ℕ}\mskip1.5mu\}\;\Varid{→}\;(\Varid{x}\;\mathbin{:}\;\Conid{A})\;\Varid{→}\;(\Varid{xs}\;\mathbin{:}\;\Conid{Vec}\;\Conid{A}\;\Varid{n})\;\Varid{→}\;\Conid{Vec}\;\Conid{A}\;(\Varid{suc}\;\Varid{n}){}\<[E]%
\ColumnHook
\end{hscode}\resethooks

Zdefiniowaliśmy już m.in. konkatenację wektorów:

\begin{hscode}\SaveRestoreHook
\column{B}{@{}>{\hspre}l<{\hspost}@{}}%
\column{10}{@{}>{\hspre}l<{\hspost}@{}}%
\column{E}{@{}>{\hspre}l<{\hspost}@{}}%
\>[B]{}\Varid{\char95 ++\char95 }\;\mathbin{:}\;\{\mskip1.5mu \Conid{A}\;\mathbin{:}\;\Conid{Set}\mskip1.5mu\}\;\Varid{→}\;\{\mskip1.5mu \Varid{n}\;\Varid{m}\;\mathbin{:}\;\Conid{ℕ}\mskip1.5mu\}\;\Varid{→}\;\Conid{Vec}\;\Conid{A}\;\Varid{n}\;\Varid{→}\;\Conid{Vec}\;\Conid{A}\;\Varid{m}\;\Varid{→}\;\Conid{Vec}\;\Conid{A}\;(\Varid{n}\;\Varid{+}\;\Varid{m}){}\<[E]%
\\
\>[B]{}[\mskip1.5mu \mskip1.5mu]\;{}\<[10]%
\>[10]{}\plus \;\Varid{v2}\;\mathrel{=}\;\Varid{v2}{}\<[E]%
\\
\>[B]{}(\Varid{x}\;\Varid{∷}\;\Varid{v1})\;\plus \;\Varid{v2}\;\mathrel{=}\;\Varid{x}\;\Varid{∷}\;(\Varid{v1}\;\plus \;\Varid{v2}){}\<[E]%
\ColumnHook
\end{hscode}\resethooks

\begin{zadanie}

Zaprogramuj funkcje vmap, która jest wektorowym odpowiednikiem map dla list.
Jaka powinna być długość wynikowego wektora?

\end{zadanie}

\begin{zadanie}

W Haskellu bardzo często używamy funkcji zip, która jest zdefiniowana następująco: \\
\\
zip :: [a] -> [b] -> [(a,b)] \\
zip (x:xs) (y:ys) = (x,y) : zip xs ys \\
zip \_ \_ = [] \\

Jak widać, przyjęto tutaj, że jeśli listy są różnej długości, to dłuższa lista jest ucinana.
Nie zawsze takie rozwiązanie jest satysfakcjonujące. Wymyśl taką sygnaturę dla funkcji zip na wektorach,
aby nie dopuścić (statycznie, za pomocą systemu typów) do niebezpiecznych wywołań.

\end{zadanie}

\begin{zadanie}

Zaprogramuj wydajną funkcję odwracającą wektor. Użyj funkcji subst, jeśli będziesz chcieć zmusić Agdę do stosowania praw arytmetyki.

\begin{hscode}\SaveRestoreHook
\ColumnHook
\end{hscode}\resethooks


\end{zadanie}

\section{Zbiory skończone - typ fin}

Przypomnijmy definicję typu fin:

\begin{hscode}\SaveRestoreHook
\column{B}{@{}>{\hspre}l<{\hspost}@{}}%
\column{3}{@{}>{\hspre}l<{\hspost}@{}}%
\column{9}{@{}>{\hspre}l<{\hspost}@{}}%
\column{E}{@{}>{\hspre}l<{\hspost}@{}}%
\>[B]{}\Keyword{data}\;\Conid{Fin}\;\mathbin{:}\;\Conid{ℕ}\;\Varid{→}\;\Conid{Set}\;\Keyword{where}{}\<[E]%
\\
\>[B]{}\hsindent{3}{}\<[3]%
\>[3]{}\Varid{zero}\;{}\<[9]%
\>[9]{}\mathbin{:}\;\{\mskip1.5mu \Varid{n}\;\mathbin{:}\;\Conid{ℕ}\mskip1.5mu\}\;\Varid{→}\;\Conid{Fin}\;(\Varid{suc}\;\Varid{n}){}\<[E]%
\\
\>[B]{}\hsindent{3}{}\<[3]%
\>[3]{}\Varid{suc}\;{}\<[9]%
\>[9]{}\mathbin{:}\;\{\mskip1.5mu \Varid{n}\;\mathbin{:}\;\Conid{ℕ}\mskip1.5mu\}\;\Varid{→}\;(\Varid{i}\;\mathbin{:}\;\Conid{Fin}\;\Varid{n})\;\Varid{→}\;\Conid{Fin}\;(\Varid{suc}\;\Varid{n}){}\<[E]%
\ColumnHook
\end{hscode}\resethooks

Dla dowolnego n ∈ ℕ typ Fin n ma dokładnie n mieszkańców (w szczególności Fin 0 jest pusty).
Własność ta sprawia, że typ Fin świetnie nadaje się do indeksowania wektorów.
Indeksowanie to jest bezpieczne, gdyż system typów wyklucza nam możliwość stworzenia indeksu, 
który wykraczałby poza dozwolony zakres.

\begin{hscode}\SaveRestoreHook
\column{B}{@{}>{\hspre}l<{\hspost}@{}}%
\column{18}{@{}>{\hspre}l<{\hspost}@{}}%
\column{E}{@{}>{\hspre}l<{\hspost}@{}}%
\>[B]{}\Varid{\char95 !\char95 }\;\mathbin{:}\;\{\mskip1.5mu \Conid{A}\;\mathbin{:}\;\Conid{Set}\mskip1.5mu\}\;\{\mskip1.5mu \Varid{n}\;\mathbin{:}\;\Conid{ℕ}\mskip1.5mu\}\;\Varid{→}\;\Conid{Vec}\;\Conid{A}\;\Varid{n}\;\Varid{→}\;\Conid{Fin}\;\Varid{n}\;\Varid{→}\;\Conid{A}{}\<[E]%
\\
\>[B]{}[\mskip1.5mu \mskip1.5mu]\;\mathbin{!}\;(){}\<[E]%
\\
\>[B]{}(\Varid{x}\;\Varid{∷}\;\Varid{xs})\;\mathbin{!}\;\Varid{zero}\;{}\<[18]%
\>[18]{}\mathrel{=}\;\Varid{x}{}\<[E]%
\\
\>[B]{}(\Varid{x}\;\Varid{∷}\;\Varid{xs})\;\mathbin{!}\;\Varid{suc}\;\Varid{i}\;\mathrel{=}\;\Varid{xs}\;\mathbin{!}\;\Varid{i}{}\<[E]%
\ColumnHook
\end{hscode}\resethooks

\begin{zadanie}

Zauważ, że typ Fin przypomina strukturą liczby naturalne, choć zawiera więcej informacji. \\
Napisz funkcję konwersji, która ''zapomina'' te dodatkowe informacje.

Np. jeśli Fin 2 = \{ 0₂, 1₂ \}, to chcemy mieć \\
toℕ 0₂ ≡ 0 \\
toℕ 1₂ ≡ 1 \\

\begin{hscode}\SaveRestoreHook
\column{B}{@{}>{\hspre}l<{\hspost}@{}}%
\column{E}{@{}>{\hspre}l<{\hspost}@{}}%
\>[B]{}\Varid{toℕ}\;\mathbin{:}\;\{\mskip1.5mu \Varid{n}\;\mathbin{:}\;\Conid{ℕ}\mskip1.5mu\}\;\Varid{→}\;\Conid{Fin}\;\Varid{n}\;\Varid{→}\;\Conid{ℕ}{}\<[E]%
\\
\>[B]{}\Varid{toℕ}\;\mathrel{=}\;\{\mskip1.5mu \mathbin{!!}\mskip1.5mu\}{}\<[E]%
\ColumnHook
\end{hscode}\resethooks

\end{zadanie}

\begin{zadanie}

Napisz funkcje, która dla danego n, zwraca największy element zbioru Fin (suc n).
Przez największy rozumiemy tutaj ten element, który jest zbudowany z największej
liczby konstruktorów.
\\
Przykładowo, dla n ≡ 3 mamy Fin 3 ≡ \{ 0₃, 1₃, 2₃ \} i jako wynik chcemy otrzymać 2₃.

\begin{hscode}\SaveRestoreHook
\column{B}{@{}>{\hspre}l<{\hspost}@{}}%
\column{E}{@{}>{\hspre}l<{\hspost}@{}}%
\>[B]{}\Varid{fmax}\;\mathbin{:}\;(\Varid{n}\;\mathbin{:}\;\Conid{ℕ})\;\Varid{→}\;\Conid{Fin}\;(\Varid{suc}\;\Varid{n}){}\<[E]%
\\
\>[B]{}\Varid{fmax}\;\mathrel{=}\;\{\mskip1.5mu \mathbin{!!}\mskip1.5mu\}{}\<[E]%
\ColumnHook
\end{hscode}\resethooks

\end{zadanie}

\begin{zadanie}

Jeśli udało Ci się zrobić poprzednie dwa zadania, to pokaż, że ich złozenie w jedną ze stron daje identyczność.
Czy potrafisz sformułować twierdzenie o złożeniu w drugą stronę?

\begin{hscode}\SaveRestoreHook
\column{B}{@{}>{\hspre}l<{\hspost}@{}}%
\column{E}{@{}>{\hspre}l<{\hspost}@{}}%
\>[B]{}\Varid{lemma-max}\;\mathbin{:}\;(\Varid{n}\;\mathbin{:}\;\Conid{ℕ})\;\Varid{→}\;\Varid{toℕ}\;(\Varid{fmax}\;\Varid{n})\;\Varid{≡}\;\Varid{n}{}\<[E]%
\\
\>[B]{}\Varid{lemma-max}\;\mathrel{=}\;\{\mskip1.5mu \mathbin{!!}\mskip1.5mu\}{}\<[E]%
\ColumnHook
\end{hscode}\resethooks

\end{zadanie}

\begin{zadanie}

W matematyce mamy \{ 0, 1, 2, .. n-1 \} ⊆ \{ 0, 1, 2, .. n-1, n \}.
Zainspirowani tym zawieraniem chcemy teraz
pokazać, że typ Fin n można osadzić w Fin (suc n). \\
Uwaga. Jednym ze sposobów byłoby po prostu użycie konstruktora suc, 
ale chcemy zrobić odwzorowanie, w którym $ k_n $ przechodzi na $ k_{n+1} $.

\begin{hscode}\SaveRestoreHook
\column{B}{@{}>{\hspre}l<{\hspost}@{}}%
\column{E}{@{}>{\hspre}l<{\hspost}@{}}%
\>[B]{}\Varid{fweak}\;\mathbin{:}\;\{\mskip1.5mu \Varid{n}\;\mathbin{:}\;\Conid{ℕ}\mskip1.5mu\}\;\Varid{→}\;\Conid{Fin}\;\Varid{n}\;\Varid{→}\;\Conid{Fin}\;(\Varid{suc}\;\Varid{n}){}\<[E]%
\\
\>[B]{}\Varid{fweak}\;\mathrel{=}\;\{\mskip1.5mu \mathbin{!!}\mskip1.5mu\}{}\<[E]%
\ColumnHook
\end{hscode}\resethooks

Po zdefiniowaniu funkcji udowodnij jej poprawność:

\begin{hscode}\SaveRestoreHook
\column{B}{@{}>{\hspre}l<{\hspost}@{}}%
\column{E}{@{}>{\hspre}l<{\hspost}@{}}%
\>[B]{}\Varid{fweak-correctness}\;\mathbin{:}\;\{\mskip1.5mu \Varid{n}\;\mathbin{:}\;\Conid{ℕ}\mskip1.5mu\}\;\Varid{→}\;(\Varid{i}\;\mathbin{:}\;\Conid{Fin}\;\Varid{n})\;\Varid{→}\;\Varid{toℕ}\;\Varid{i}\;\Varid{≡}\;\Varid{toℕ}\;(\Varid{fweak}\;\Varid{i}){}\<[E]%
\\
\>[B]{}\Varid{fweak-correctness}\;\mathrel{=}\;\{\mskip1.5mu \mathbin{!!}\mskip1.5mu\}{}\<[E]%
\ColumnHook
\end{hscode}\resethooks


\end{zadanie}

\begin{zadanie}
Pokaż, że dla każdego n ∈ ℕ, możemy stablicować wszystkie elementy
typu Fin n w n-elementowym wektorze.

\begin{hscode}\SaveRestoreHook
\column{B}{@{}>{\hspre}l<{\hspost}@{}}%
\column{E}{@{}>{\hspre}l<{\hspost}@{}}%
\>[B]{}\Varid{tabFins}\;\mathbin{:}\;(\Varid{n}\;\mathbin{:}\;\Conid{ℕ})\;\Varid{→}\;\Conid{Vec}\;(\Conid{Fin}\;\Varid{n})\;\Varid{n}{}\<[E]%
\\
\>[B]{}\Varid{tabFins}\;\mathrel{=}\;\{\mskip1.5mu \mathbin{!!}\mskip1.5mu\}{}\<[E]%
\ColumnHook
\end{hscode}\resethooks

\end{zadanie}

\end{document}
