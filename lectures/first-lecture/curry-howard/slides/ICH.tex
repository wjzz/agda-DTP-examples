\documentclass{beamer}

% wygląd slajdów 
% \usetheme{Warsaw} 
\usetheme{Boadilla}

% \usecolortheme{seahorse} 

% brak tej dziwnej nawigacji w prawym dolnym rogu 
\setbeamertemplate{navigation symbols}{} 

% brak zbędnych informacji w stopce
\setbeamertemplate{footline}[page number]{}

% klikalne linki
\usepackage{hyperref}

% kodowanie utf8 
%\usepackage[OT4]{fontenc} 
%\usepackage[T1]{fontenc}
\usepackage[utf8]{inputenc} 
\usepackage{polski}
\usepackage{listings}
\usepackage{amsthm}
\usepackage{semantic}

\newtheorem{thm}{Twierdzenie}

% konfiguracja niektórych parametrów 
% \setbeamercovered{transparent} 
\setbeamertemplate{bibliography item}[text] 
\setlength{\parskip}{1ex} 
\setlength{\parindent}{0pt} 

\title{Seminarium: Programowanie w teorii typów}
\subtitle{Teoria typów}
\author{Wojciech Jedynak, Paweł Wieczorek} 
\institute{Instytut Informatyki Uniwersytetu Wrocławskiego}

\AtBeginSection[]
{
   \begin{frame}
       \frametitle{}
       \tableofcontents[sectionstyle=show/hide, subsectionstyle=show/show/hide]
   \end{frame}
}

\begin{document}

%\lstset{language=Lisp}

\maketitle

%%%%%%%%%%%%%%%%%%%%%%%%

% \tableofcontents[hidesubsections]

%%%%%%%%%%%%%%%%%%%%%%%%%
\section{Matematyka konstruktywna}

\begin{frame}
\frametitle{Matematyka konstruktywna}

\begin{itemize}
 \item powstały na początku poprzedniego wieku pogląd na temat fundamentów matematyki
 \item L.E.J.Brouwer, twórca ideologii
 \item empiryczna zawartość twierdzeń matematycznych
 \item co znaczy orzeczenie o istnieniu pewnego obiektu?
 \item odrzucenie dowodów przez sprowadzenie do sprzeczności
 \item odrzucenie idealistycznego podejścia do prawdziwości orzeczeń
 \item E. Bishop, konstruktywna analiza matematyczna
\end{itemize}


\end{frame}

%%%%%%%%%%%%%%%%%%%%%%%%%

\begin{frame}
\frametitle{Książkowy przykład twierdzenia niekonstruktywnego}

\begin{thm}
 Istnieją takie dwie liczby niewymierne $a$ oraz $b$, że $a^b$ jest liczbą wymierną.
\end{thm}

\begin{proof}
 Orzeczenie, że $\sqrt{2}^{\sqrt{2}} \in \mathbf{Q}$ musi być prawdziwe lub musi być fałszywe.
\begin{itemize}
 \item jeżeli jest prawdziwe to mamy szukane $a$ oraz $b$
 \item jeżeli jest fałszywe to niech $a = \sqrt{2}^{\sqrt{2}}$ oraz $b = \sqrt{2}$, wtedy $a^b = 2$
\end{itemize}

\end{proof}

Jedyne co wiemy, to to że muszą istnieć takie liczby. 

\end{frame}

%%%%%%%%%%%%%%%%%%%%%%%%%

\begin{frame}
\frametitle{Kolejny przykład.}

\begin{thm}[klasycznie]
 Jeżeli funkcja $f$ jest ciągła na przedziale $[0, 1]$ oraz wartości funkcji na~krańcach przedziału mają różne znaki
to istnieje punkt w~tym przedziale na~którym funkcja się zeruje.
\end{thm}

\begin{thm}[konstruktywnie]
 Jeżeli funkcja $f$ jest ciągła na przedziale $[0, 1]$ oraz wartości funkcji na~krańcach przedziału mają różne znaki
to dla każdego $\epsilon > 0$ istnieje punkt w~tym przedziale na~którym bezwzględna wartość funkcji jest mniejsza od
$\epsilon$.
\end{thm}

\end{frame}

%%%%%%%%%%%%%%%%%%%%%%%%%


\begin{frame}
\frametitle{Interpretacja Brouwer-Heyting-Kołmogorow}

\begin{itemize}
 \item $A \wedge B$ to konstrukcja składająca się z~dwóch pod-konstrukcji
 \item $A \vee B$ to konstrukcja składająca się z lewej lub prawej pod-konstrukcji
 \item $A \to B$ to metoda przekształcająca konstrukcję $B$ mając do~dyspozycji $A$
 \item $\bot$ absurd, konstrukcja której nie można zrealizować
 \item $\forall x. P(x)$ to metoda przekształcająca wartość $a$ w~konstrukcję $P(a)$
 \item $\exists x. P(x)$ to konstrukcja mająca składać się ze świadka $a$ oraz z konstrukcji $P(a)$
 \item $\neg A$ to skrót od $A \to \bot$ 
 \item Czy przy tej interpretacji wszystkie klasyczne prawa mają sens?
\begin{itemize}
\item $\exists x\;P(x) \vee \neg \exists x\;P(x)$
\item $(\neg \forall x\;\neg P(x) ) \to \exists x\;P(x)$
\end{itemize}

\end{itemize}


\end{frame}

%%%%%%%%%%%%%%%%%%%%%%%%%

\begin{frame}
\frametitle{System naturalnej dedukcji}

\begin{itemize}
 \item system dowodzenia
 \item posługujemy się sądami $\Gamma |- \varphi$
 \item dowód to wyprowadzenie o~strukturze drzewa
 \begin{itemize}
 \item korzeń - wniosek (sąd)
 \item węzeł - reguła wnioskowania
 \item liść - aksjomat
 \end{itemize}
 \item reguły wprowadzania i eliminacji spójników logicznych
\end{itemize}


\end{frame}

%%%%%%%%%%%%%%%%%%%%%%%%%

\begin{frame}
\frametitle{System naturalnej dedukcji}

\begin{center}
\begin{tabular}{lr}
\inference[I$\wedge$]{
\Gamma |- A \qquad \Gamma |- B
}
{
\Gamma |- A \wedge B
}
&
\inference[E$\wedge_1$]{
\Gamma |- A \wedge B
}
{
\Gamma |- A 
}
\end{tabular}
\end{center}

\begin{center}
\begin{tabular}{lr}

\inference[I$\vee_1$]{
\Gamma |- A 
}
{
\Gamma |- A \vee B
}
&
\inference[E$\vee$]{
\Gamma |- A \vee B \qquad \Gamma, A |- C \qquad \Gamma, B |- C
}
{
\Gamma |- C
}

\end{tabular}
\end{center}

\begin{center}
\begin{tabular}{lr}

\inference[I$\to$]{
\Gamma, A |- B
}
{
\Gamma |- A \to B
}
&
\inference[E$\to$]{
\Gamma |- A \to B \qquad \Gamma |- A
}
{
\Gamma |- B
}

\end{tabular}
\end{center}


\begin{center}
\begin{tabular}{c}
\inference[AX]{
}
{
\Gamma, A |- A 
}
\end{tabular}
\end{center}

\end{frame}

%%%%%%%%%%%%%%%%%%%%%%%%%

%%%%%%%%%%%%%%%%%%%%%%%%%
\section{Teoria typów Martina-L\"{o}fa}

\begin{frame}
\frametitle{Historia, idee, początki}

\begin{itemize}
 \item Teoria typów jako logika matematyczna - B. Russel
 \[
  A = \{ w \mid w \not\in w \}
 \]
 \item $\lambda$-rachunek , funkcja jako pojęcie pierwotne  - A.Church
 \item System typów, likwidacja paradoksu Kleene'go
\[
 K = \lambda x.\; \neg (x\; x)
\]

\[
 (K\; K) = \neg (K\; K) = \neg \neg (K\; K) = \cdots
\]

\end{itemize}


\end{frame}

%%%%%%%%%%%%%%%%%%%%%%%%%

\begin{frame}
\frametitle{Wzbogacony system typów o więcej sądów}

\begin{itemize}
 \item Teoria typów Martina L\"{o}f'a - system typów dla $\lambda$-rachunku, w którym możemy wydawać różne sądy:
       
\begin{itemize}
 \item $A$\;set -  jest zbiorem
 \item $a \in A$ - $a$ jest elementem zbioru
 \item $a =_A b \in A$ - $a$ oraz $b$ są równymi elementami w~zbiorze A
 \item $A = B$ - $A$ oraz $B$ są równymi zbiorami
\end{itemize}

\item elementy zbiorów dzielimy na
\begin{itemize}
 \item kanoniczne - wartości (postać normalna)
 \item niekanoniczne - obliczenia 
\end{itemize}


\item sformułowanie zbioru to
\begin{itemize}
 \item określenie kanonicznych elementów jakie ten zbiór zawiera
 \item określenie co znaczy że dwa elementy są równe w~tym zbiorze
 \item określenie obliczeń
\end{itemize}

\end{itemize}


\end{frame}

%%%%%%%%%%%%%%%%%%%%%%%%%

\begin{frame}
\frametitle{Liczby naturalne}

\begin{center}
\begin{tabular}{lcr}
\inference{
}
{
Nat\; set
}
&
\inference{
}
{
0 \in Nat
}
&
\inference{
n \in Nat
}
{
succ(n) \in Nat
}
\end{tabular}
\end{center}


\begin{center}
\begin{tabular}{lr}
\inference{
n = n' \in Nat
}
{
succ(n) = succ(n') \in Nat
}
&
\inference{
A\; set \qquad n \in Nat  \\
z \in A \qquad s \in Nat \to A \to A 
}
{
natrec(n, z, s) \in A
}
\end{tabular}
\end{center}

\begin{center}
\begin{tabular}{c}
\inference{
A\; set \qquad  \\
z \in A \qquad s \in Nat \to A \to A 
}
{
natrec(0, z, s) = z \in A
}
\end{tabular}
\end{center}

\begin{center}
\begin{tabular}{c}
\inference{
A\; set \qquad n \in Nat  \\
z \in A \qquad s \in Nat \to A \to A 
}
{
natrec(succ(n), z, s) = s(n, (natrec\;(n, z, s)) \in A
}
\end{tabular}
\end{center}

\end{frame}

%%%%%%%%%%%%%%%%%%%%%%%

\begin{frame}
\frametitle{Przykład iloczyn kartezjański w~uproszczonej formie}

\begin{center}
\begin{tabular}{lcr}
\inference{
A\; set \qquad B\; set
}
{
A \times B\; set
}
&
\inference{
A = A' \qquad B = B'
}
{
A \times B = A' \times B'
}
&
\inference{
a \in A \qquad b \in B
}
{
(a,b) \in A \times B
}
\end{tabular}
\end{center}


\begin{center}
\begin{tabular}{c}
\inference{
a = a' \in A \qquad b = b' \in B
}
{
(a,b) = (a', b') \in A \times B
}
\end{tabular}
\end{center}

\begin{center}
\begin{tabular}{c}
\inference{
C\;set \qquad p \in A \times B \qquad f(x, y) \in C\; [x \in A,\; y \in B]
}
{
split(p, f) \in C
}
\end{tabular}
\end{center}

\begin{center}
\begin{tabular}{c}
\inference{
C\;set \qquad a \in A \qquad b \in B \qquad f(x, y) \in C\; [x \in A,\; y \in B]
}
{
split((a,b), f) = f(a,b) \in C
}
\end{tabular}
\end{center}

\end{frame}

%%%%%%%%%%%%%%%%%%%%%%%%%

%%%%%%%%%%%%%%%%%%%%%%%%%

\begin{frame}
\frametitle{Izomorfizm Curry-Howard (proposition as types)}

\begin{itemize}
 \item Martin-L\"{o}f, Curry, Howard, deBruijn i wiele innych
 \item typy oznaczają formuły
 \item otypowane termy oznaczają dowody dla swoich typów (formuł)
 \item izomorfizm pomiędzy wyprowadzeniami formuł w logice intuicjonistycznej a
       sądami w~systemie typów
 \item realizacja BHK
\end{itemize}

\begin{center}
\begin{tabular}{lr}

\inference[I$\to$]{
\Gamma, x : A |- t : B
}
{
\Gamma |- \lambda x:A. t : A \to B
}
&
\inference[E$\to$]{
\Gamma |- f : A \to B \qquad \Gamma |- x : A
}
{
\Gamma |- f \; x : B
}

\end{tabular}
\end{center}


\end{frame}

%%%%%%%%%%%%%%%%%%%%%%%%%

\begin{frame}
\frametitle{System naturalnej dedukcji}

\begin{center}
\begin{tabular}{lr}
\inference[I$\wedge$]{
\Gamma |- M : A \qquad \Gamma |- N : B
}
{
\Gamma |- (M, N) : A \wedge B
}
&
\inference[E$\wedge_1$]{
\Gamma |- M : A \wedge B
}
{
\Gamma |- fst\;M : A 
}
\end{tabular}
\end{center}

\begin{center}
\begin{tabular}{c}

\inference[I$\vee_1$]{
\Gamma |- M : A 
}
{
\Gamma |- inl\; M : A \vee B
}
\end{tabular}
\end{center}

\begin{center}
\begin{tabular}{c}

\inference[E$\vee$]{
\Gamma |- M : A \vee B \qquad \Gamma, x : A |- P : C \qquad \Gamma, x : B |- Q : C
}
{
\Gamma |- when\;M\;(\lambda x.P)\;(\lambda x.Q) : C
}

\end{tabular}
\end{center}

\begin{center}
\begin{tabular}{lr}

\inference[I$\to$]{
\Gamma, A |- M : B
}
{
\Gamma |- \lambda x. M : A \to B
}
&
\inference[E$\to$]{
\Gamma |- M : A \to B \qquad \Gamma |- N A
}
{
\Gamma |- M\; N : B
}

\end{tabular}
\end{center}


\begin{center}
\begin{tabular}{c}
\inference[AX]{
}
{
\Gamma, x : A |- x : A 
}
\end{tabular}
\end{center}

\end{frame}

%%%%%%%%%%%%%%%%%%%%%%%%%

\begin{frame}
\frametitle{Izomorfizm Curry-Howard (proposition as types)}

\begin{itemize}
 \item fundamentalna teoria według kryteriów matematyki konstruktywnej
 \item pojęciem pierwotnym jest funkcja, nie zbiór
 \item nie używamy klasycznych definicji pojęć, mogą być one nie konstruktywne, nie dające się zrealizować
 \item funkcje które definiujemy są obliczalne i totalne
 \item teoria nie wyrażona jako FOL, lecz kodująca ją w~sobie
\end{itemize}
\end{frame}

%%%%%%%%%%%%%%%%%%%%%%%%%

\begin{frame}
\frametitle{Sądy mają więcej interpretacji}

\begin{itemize}
 \item $A$\;set -  jest zbiorem
 \item $A$\;set -  jest problemem, zagadnieniem, zadaniem
 \item $A$\;prop - jest formułą logiczną
 \item $A$\;true - umiemy zrealizować $A$, istnieje dowód $A$
 \item $a \in A$ - $a$ jest elementem zbioru
 \item $a \in A$ - $a$ jest dowodem propozycji $A$
 \item $a \in A$ - $a$ jest programem spełniającym specyfikację $A$
 \item $a \in A$ - $a$ jest rozwiązaniem problemu $A$
\end{itemize}
\end{frame}

%%%%%%%%%%%%%%%%%%%%%%%%%


\end{document}
