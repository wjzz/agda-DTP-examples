\documentclass{beamer}

\usetheme{Boadilla}

% brak tej dziwnej nawigacji w prawym dolnym rogu 
\setbeamertemplate{navigation symbols}{} 

% brak zbędnych informacji w stopce
\setbeamertemplate{footline}[page number]{}

% klikalne linki
\usepackage{hyperref}

% kodowanie utf8 
\usepackage[utf8]{inputenc} 
\usepackage{polski}
\usepackage{amsthm}
\usepackage{semantic}

\newtheorem{thm}{Twierdzenie}

% konfiguracja niektórych parametrów 
% \setbeamercovered{transparent} 
\setbeamertemplate{bibliography item}[text] 
\setlength{\parskip}{1ex} 
\setlength{\parindent}{0pt} 

\title{Seminarium: Programowanie w teorii typów}
\subtitle{Teoria typów cd.}
\author{Wojciech Jedynak, Paweł Wieczorek} 
\institute{Instytut Informatyki Uniwersytetu Wrocławskiego}

%\AtBeginSection[]
%{
   %\begin{frame}
       %\frametitle{}
       %\tableofcontents[sectionstyle=show/hide, subsectionstyle=show/show/hide]
   %\end{frame}
%}

\begin{document}

%\lstset{language=Lisp}

\maketitle

%%%%%%%%%%%%%%%%%%%%%%%%

% \tableofcontents[hidesubsections]

%%%%%%%%%%%%%%%%%%%%%%%%%

\begin{frame}
\frametitle{Plan na dziś}

\begin{itemize}
 \item Co już umiemy:
 \begin{itemize}
 \item posługiwać się systemem naturalnej dedukcji
 \item przerbnęliśmy przez regułki dla $\Pi$-typów oraz $\Sigma$-typów
 \item zobaczyliśmy jak typy kodują kwantyfikatory z logiki pierwszego rzędu
 \end{itemize}

 \item Co dziś zobaczymy:
 \begin{itemize}
 \item typy wyliczeniowe
 \item równość w teorii typów
 \item W-typy
 \item uniwersa
 \end{itemize}

\end{itemize}

\end{frame}

%%%%%%%%%%%%%%%%%%%%%%%%%

\begin{frame}
\frametitle{Typy wyliczeniowe}

\begin{center}
\begin{tabular}{lr}
\inference{
}
{
N_k\; set
}
&
\inference{
\mbox{dla $m < k$}
}
{
m_k \in N_k
}
\end{tabular}
\end{center}


\begin{center}
\begin{tabular}{c}
\inference{
a \in N_k \qquad C(v)\;set\;[v \in N_k] \\
c_0 \in C(0_k) \\
\vdots \\
c_{k-1} \in C((k-1)_k)
}
{
case_k(a, c_0, \cdots, c_{k-1}) \in C(a)
}
\end{tabular}
\end{center}


\begin{center}
\begin{tabular}{c}
\inference{
C(v)\;set\;[v \in N_k] \\
c_0 \in C(0_k) \\
\vdots \\
c_{k-1} \in C((k-1)_k)
}
{
case_k(m_k, c_0, \cdots, c_{k-1}) = c_m\in C(m_k)
}
\end{tabular}
\end{center}

\end{frame}

%%%%%%%%%%%%%%%%%%%%%%%%%

\begin{frame}
\frametitle{Typy wyliczeniowe}

\begin{eqnarray*}
 N_0 &=& \bot \\
 \neg A &=& A \to N_0 \\
\pause
 N_1 &=& \top \\
 N_1 &=& Unit \\
 0_1 &=& tt \\
\pause
 N_2 &=& Bool \\
 0_2 &=& true \\
 1_2 &=& false \\
 case_2(a,c_0, c_1) &=& if\;a\; then\;c_0\;else\;c_1
\end{eqnarray*}

\pause
\begin{center}
\begin{tabular}{lr}
\inference{
a \in N_0 \qquad C(v)\;set\;[v \in N_0]
}
{
case_0(a) \in C(a)
}
&
\inference{
\bot \;true \qquad C\;prop
}
{
C\;true
}
\end{tabular}
\end{center}

\end{frame}

%%%%%%%%%%%%%%%%%%%%%%%%%

\begin{frame}
\frametitle{Wyrażenia}

\begin{itemize}
\item język wyrażeń:
\begin{itemize}
\item $e(e')$ - aplikacja ( $e(e_1, \cdots, e_n)$ skrócony zapis $e(e_1)\cdots(e_n)$ )
\item $(x)e$ - abstrakcja
\item stałe, np $\lambda$, $\Pi$, $apply, 0, succ$
\end{itemize}

\end{itemize}

\pause

\begin{itemize}
\item posługujem się prostym systemem typów (z~jednym typem bazowym $O$) zwanym ,,arnościami''
\item $e(e') : \beta$ gdy $e : \alpha \to \beta$, $e' : \alpha$
\item konstruktory:
\begin{itemize}
\item $\lambda : (O \to O) \to O$
\item $\Pi: O \to (O \to O) \to O$
\item $apply : O\to O \to O$
\end{itemize}

\item otrzymujemy silną normalizację oraz rozstrzygalność dla równości definicyjnej
\end{itemize}


\end{frame}

%%%%%%%%%%%%%%%%%%%%%%%%%
%%%%%%%%%%%%%%%%%%%%%%%%%


\begin{frame}
\frametitle{Typ identycznościowy}

\begin{itemize}
\item Sądy a formuły to nie to samo
\begin{itemize}
\item nie możemy powiedzieć wewnątrz systemu ,,jeżeli t ma typ A to ...'': $\Gamma |- (t \in A) -> \varphi$
\item nie możemy też powiedzieć wewnątrz systemu ,,dany sąd jest nieprawdziwy'': $\Gamma |- \neg (a = b \in A)$
\end{itemize}

\pause
\item Co ma oznaczać stwierdzenie, że elementy $a$ oraz $b$ są sobie równe?
\begin{itemize}
\item nie poróżnia je żadna własność (równość Leibniza)
\item dla każdego predykatu $P$: $\forall x\; y.\;  x=y \to P(x) \to P(y)$
\end{itemize}
\pause
\item Co znaczy, że funkcje są sobie równe?
\begin{itemize}
\item Ile jest funkcji (jako zbiór par) sortujących? Ekstensjonalnie tylko jedna.
\item $(\forall xs \in List.\,qsort(xs) \equiv isort(xs)) \to qsort \equiv isort$
\end{itemize}
\pause
\item Równość jest formułą atomową, chcemy mieć dla niej typ.
\begin{itemize}
\item który zgodnie z \emph{formulas-as-types} jest zamieszkany wtedy i tylko wtedy, gdy elementy są sobie równe
\end{itemize}

\end{itemize}

\end{frame}

%%%%%%%%%%%%%%%%%%%%%%%%%

\begin{frame}
\frametitle{Typ identycznościowy}

\begin{itemize}
 \item mamy:

\begin{itemize}
 \item równość definicyjna: $e \equiv e'$
 \item równość elementów o tym samym typie $a = b \in A$
 \item równosć typów $A = B$
\end{itemize}

 \item chcemy jeszcze:
\begin{itemize}
 \item formuła logiczna $[a =_A b]$
\end{itemize}

\end{itemize}

\end{frame}

%%%%%%%%%%%%%%%%%%%%%%%%%

\begin{frame}
\frametitle{Typ identycznościowy}

\begin{center}
\begin{tabular}{lr}
\inference{
A\;set\qquad a\in A \qquad b \in A
}
{
[a =_A b]\; set
}
&
\inference{
a \in A
}
{
id(a) \in [a =_A a]
}
\end{tabular}
\end{center}

\begin{center}
\begin{tabular}{c}
\inference{
a = a' \in A \qquad b = b' \in A
}
{
[a =_A b] = [a' =_A b']
}
\end{tabular}
\end{center}

\pause 
\begin{center}
\begin{tabular}{c}
\inference{
a = b \in A
}
{
id(a) \in [a =_A b]
}
\end{tabular}
\end{center}

\pause 
\begin{center}
\begin{tabular}{c}
\inference{
\dfrac{}{id(a) \in [a =_A a]} \qquad \dfrac{a = a \in A \quad a = b \in A}{[a =_A a] = [a =_A b]}
}
{
id(a) \in [a =_A b]
}
\end{tabular}
\end{center}

\end{frame}


%%%%%%%%%%%%%%%%%%%%%%%%%

\begin{frame}
\frametitle{Typ identycznościowy}

\begin{center}
\begin{tabular}{c}
\inference{
a \in A \qquad b \in A \qquad c \in [a =_A b]
}
{
a = b \in A
}
\end{tabular}
\end{center}

\begin{itemize}
 \item silna eliminacja, równość ekstensjonalna
 \item chcemy by system nieodróżniał od siebie elementów o których możemy wydać sąd równościowy
 \item ,,gubimy dowód'' $[a =_A b]$, a więc system chcąc sprawdzić $a = b \in A$ może być zmuszony by go odgadnąć
 \item nierozstrzygalność
\end{itemize}
\end{frame}


%%%%%%%%%%%%%%%%%%%%%%%%%

\begin{frame}
\frametitle{Typ identycznościowy}

\begin{center}
\begin{tabular}{c}
\inference{
a \in A \qquad b \in A \qquad c \in [a =_A b] \\
C(x,y,z)\;set\;[x \in A, y \in A, z \in [x =_A y]] \\
d(x) \in C(x,x,id(x))\;[x \in A]
}
{
idpeel(c,d) \in C(a,b,c)
}
\end{tabular}
\end{center}


\begin{center}
\begin{tabular}{c}
\inference{
a \in A \\
C(x,y,z)\;set\;[x \in A, y \in A, z \in [x =_A y]] \\
d(x) \in C(x,x,id(x))\;[x \in A]
}
{
idpeel(id(a),d) = d(a) \in C(a,a,id(a))
}
\end{tabular}
\end{center}

\begin{itemize}
 \item słaba eliminacja
 \item zasada indukcji jak dla innych typów
 \item zachowujemy rozstrzygalność dla równości definicyjnej wyrażeń oraz sądów równościowych
\end{itemize}

\end{frame}



%%%%%%%%%%%%%%%%%%%%%%%%%

\begin{frame}
\frametitle{Typ identycznościowy}

\begin{itemize}
 \item niech $c \in [a =_A b]$.
 \item niech $C(x,y,z) \equiv [y =_A x]$

\begin{center}
\begin{tabular}{c}
\inference{
a \in A \qquad b \in A \qquad c \in [a =_A b] \\
C(x,y,z)\;set\;[x \in A, y \in A, z \in [x =_A y]] \\
id(x) \in C(x,x,id(x))\;[x \in A]
}
{
idpeel(c,id) \in C(a,b,c)
}
\end{tabular}
\end{center}

\begin{center}
\begin{tabular}{c}
\inference{
a \in A \qquad b \in A \qquad c \in [a =_A b] \\
[y =_A x]\;set\;[x \in A, y \in A, z \in [x =_A y]] \\
id(x) \in [x =_A x]\;[x \in A]
}
{
idpeel(c,id) \in [b =_A a]
}
\end{tabular}
\end{center}
\item $symm(c) \equiv idpeel(c, id)$
\end{itemize}

\end{frame}

%%%%%%%%%%%%%%%%%%%%%%%%%

%%%%%%%%%%%%%%%%%%%%%%%%%

\begin{frame}
\frametitle{Typ identycznościowy}

\begin{itemize}
 \item niech $e \in [a =_A b]$, $e' \in [b =_A c]$.
 \item niech $C(x,y,z) \equiv [y =_A c] \to [x =_A c]$

\begin{center}
\begin{tabular}{c}
\inference{
a \in A \qquad b \in A \qquad e \in [a =_A b] \\
C(x,y,z)\;set\;[x \in A, y \in A, z \in [x =_A y]] \\
\lambda y. y \in C(x,x,id(x))\;[x \in A]
}
{
idpeel(e,(x)\lambda y. y) \in C(a,b,c)
}
\end{tabular}
\end{center}

\begin{center}
\begin{tabular}{c}
\inference{
a \in A \qquad b \in A \qquad e \in [a =_A b] \\
[y =_A c] \to [x =_A c]\;set\;[x \in A, y \in A, z \in [x =_A y]] \\
\lambda y. y \in \;[x =_A c] \to [x =_A c] [x \in A]
}
{
idpeel(e,(x)\lambda y. y) \in [b =_A c] \to [a =_A c]
}
\end{tabular}
\end{center}
\item $trans(e, e') \equiv apply(idpeel(d, (x)\lambda y. y), e')$
\end{itemize}

\end{frame}


%%%%%%%%%%%%%%%%%%%%%%%%%
%%%%%%%%%%%%%%%%%%%%%%%%%
%%%%%%%%%%%%%%%%%%%%%%%%%

\begin{frame}
\frametitle{Typ identycznościowy}

\begin{center}
\begin{tabular}{c}
\inference{
P(x)\;prop\;[x \in A]\qquad a \in A \qquad b \in A \qquad [a =_A b]\;true \qquad P(a)\;true
}
{
P(b)\; true
}
\end{tabular}
\end{center}

\pause 

\begin{center}
\begin{tabular}{c}
\inference{
P(x)\;set\;[x \in A]\qquad a \in A \qquad b \in A \qquad c \in [a =_A b] \qquad p \in P(a)
}
{
subst(c,p) \in P(b)
}
\end{tabular}
\end{center}


\begin{itemize}
 \item rowność Leibniza
 \item ,,kontrolowane'' podstawienie
\end{itemize}

\end{frame}

%%%%%%%%%%%%%%%%%%%%%%%%%
%%%%%%%%%%%%%%%%%%%%%%%%%

\begin{frame}

\frametitle{Typ identycznościowy}

\begin{itemize}
 \item niech $a \in A$, $b \in A$, $e \in [a =_A b]$
 \item niech $P(x)\;set\;[x \in A]$, $p \in P(a)$
 \item niech $C(x,y,z) \equiv P(x) \to P(y)$

\begin{center}
\begin{tabular}{c}
\inference{
a \in A \qquad b \in A \qquad e \in [a =_A b] \\
C(x,y,z)\;set\;[x \in A, y \in A, z \in [x =_A y]] \\
id(x) \in C(x,x,id(x))\;[x \in A]
}
{
idpeel(e,(x)\lambda y. y) \in C(a,b,c)
}
\end{tabular}
\end{center}

\begin{center}
\begin{tabular}{c}
\inference{
a \in A \qquad b \in A \qquad e \in [a =_A b] \\
P(x) \to P(y)\;set\;[x \in A, y \in A, z \in [x =_A y]] \\
\lambda y. y \in P(x) \to P(x)\;[x \in A]
}
{
idpeel(e,(x)\lambda y. y) \in P(a) \to P(b)
}
\end{tabular}
\end{center}

\item $subst(e,p) \equiv apply(idpeel(e,(x)\lambda y.y), p)$
\end{itemize}


\end{frame}


%%%%%%%%%%%%%%%%%%%%%%%%%
%%%%%%%%%%%%%%%%%%%%%%%%%

\begin{frame}

\frametitle{Typ identycznościowy}

\begin{itemize}
 \item Nie potrafimy wydać sądu $if\;b\;then\;c\;else\;c = c \in A\;[b \in Bool, c \in A]$
 \item Potrafimy udowodnić $(\Pi b \in Bool)(\Pi c \in A)[ if\;b\;then\;c\;else\;c =_A c ]$
\end{itemize}

\end{frame}

%%%%%%%%%%%%%%%%%%%%%%%%%
%%%%%%%%%%%%%%%%%%%%%%%%%

\begin{frame}
\frametitle{Uniwersum}

\begin{center}
\begin{tabular}{lr}
\inference{
}
{
Set\;set
}
&
\inference{
}
{
Nat \in Set
}
\end{tabular}
\end{center}

\begin{center}
\begin{tabular}{lr}
\inference{
A \in Set \qquad B \in Set
}
{
A \to B \in Set
}
&
\inference{
A \in Set \qquad B \in Set
}
{
A \times B \in Set
}
\end{tabular}
\end{center}

\[ \cdots \]

\begin{center}
\begin{tabular}{c}
\inference{
}
{
Set \in Set
}
\end{tabular}
\end{center}


\end{frame}

%%%%%%%%%%%%%%%%%%%%%%%%%

\begin{frame}
\frametitle{Uniwersum ala Tarski}

\begin{itemize}
 \item posługujemy się nazwami(kodami, identyfikatorami, ...) typów, a nie typami bezpośrednio
 \item nazwy są zwykłymi danymi
\end{itemize}

\end{frame}

%%%%%%%%%%%%%%%%%%%%%%%%%

\begin{frame}
\frametitle{Uniwersum ala Tarski}

\begin{center}
\begin{tabular}{lr}
\inference{
}
{
U\;set
}
&
\inference{
}
{
\widehat{Nat}\in U
}
\end{tabular}
\end{center}

\begin{center}
\begin{tabular}{lcr}
\inference{
}
{
\widehat{N_0}\in U
}
&
\inference{
}
{
\widehat{N_1}\in U
}
&
\inference{
}
{
\widehat{N_2}\in U
}
\end{tabular}
\end{center}

\[ \cdots \]

\begin{center}
\begin{tabular}{lr}
\inference{
a\in U\qquad b \in U
}
{
\widehat{a \to b} \in U
}
&
\inference{
a\in U\qquad b \in U
}
{
\widehat{a \times b}\in U
}
\end{tabular}
\end{center}

\end{frame}

%%%%%%%%%%%%%%%%%%%%%%%%%

\begin{frame}
\frametitle{Uniwersum ala Tarski}

\begin{itemize}
 \item Funkcja semantyczna $Set$ bierze nazwę typu a zwraca typ
\end{itemize}

\begin{center}
\begin{tabular}{c}
\inference{
u \in U
}
{
Set(u)\;set
}
\end{tabular}
\end{center}

\begin{eqnarray*}
 Set(\widehat{Nat}) &=& Nat \\
 Set(\widehat{N_0}) &=& N_0 \\
 Set(\widehat{N_1}) &=& N_1 \\
 Set(\widehat{N_2}) &=& N_2 \\
 &...& \\
 Set(\widehat{a\to b}) &=& Set(a) \to Set(b) \\
 Set(\widehat{a \times b}) &=& Set(a) \times Set(b) \\
\end{eqnarray*}


\end{frame}

%%%%%%%%%%%%%%%%%%%%%%%%%%%%

\begin{frame}
\frametitle{Uniwersum ala Tarski}

\begin{itemize}
 \item Co nam się nie podoba w~tej funkcji?
\begin{eqnarray*}
 g    &\in & Nat \to Bool \cup Nat \\
 g(2n) &=& left(true) \\
 g(2n+1) &=& right(42)
\end{eqnarray*}
\pause
 \item Bezpieczniejsza wersja
\begin{eqnarray*}
 B_{2n}   &=& Bool \\
 B_{2n+1} &=& Nat  \\
 f    &\in & \prod_{n \in N} B_n \\
 f(2n)    &=& true \\
 f(2n+1)  &=& 42 
\end{eqnarray*}
\end{itemize}

\end{frame}

%%%%%%%%%%%%%%%%%%%%%%%%%%%%

\begin{frame}
\frametitle{Uniwersum ala Tarski}

\begin{eqnarray*}
B(x)  &\equiv  & Set( case_2(apply(isEven, x), \widehat{Bool}, \widehat{Nat}))  \\
b     & \equiv & (x) case_2(apply(isEven, x), true, 42) \\
\pause
\lambda x . b & \in & (\Pi n \in Nat) B(n) \\
apply(\lambda x . b, n) & \in & B(n)\; [ n \in Nat ] 
\end{eqnarray*}

\pause

\begin{itemize}
 \item niech $negb \in Bool \to Bool$.
\end{itemize}

\begin{center}
\begin{tabular}{c}
\inference
{
\dfrac{}
{
negb \in Bool \to Bool
}
\;\;
\dfrac{
apply(\lambda x . b, 2) \in B(2) \qquad B(2) = Bool
}
{
apply(\lambda x . b, 2) \in Bool
}
}
{
apply(negb, apply(\lambda x . b, 2)) \in Bool 
}
\end{tabular}
\end{center}

\end{frame}

%%%%%%%%%%%%%%%%%%%%%%%%%%%%



\begin{frame}[fragile]
\frametitle{Uniwersum ala Tarski}

\begin{itemize}
 \item $nAry(A,n)$ = $\underbrace{A \to A \to \cdots A}_{n+1}$

\begin{eqnarray*}
nary  &\equiv  & (a,x) natrec(x, a, (n,y) \widehat{a \to y})  \\
\lambda x. nary(a,x) & \in & (\Pi x \in Nat) U\; [a \in U] \\
\lambda a. \lambda x. nary(a,x) & \in & (\Pi a \in U) (\Pi x \in Nat) U \\
nAry &\equiv& \lambda a. \lambda x. nary(a,x) \\
\\\pause
boolfun  & \equiv & (n)Set(apply(apply(nAry, \widehat{Bool}), n)) \\
\lambda n. boolfun & \in & (\Pi n \in Nat) Set 
\end{eqnarray*}

\item podobnie moglibyśmy generalizować proste schematy dla krotek $A^2$, $A^3$, itd:
\begin{verbatim}
 fst (a,b) = a   fst3(a,b,c) = a   fst4(a,b,c,d) = a
\end{verbatim}

\end{itemize}


\end{frame}

%%%%%%%%%%%%%%%%%%%%%%%%%

\begin{frame}
\frametitle{Uniwersum ala Tarski}

\begin{itemize}
 \item Wyrażenie kodu dla $A \to B$ jest proste.
~ 


\begin{center}
\begin{tabular}{lr}
\inference{
a \in U \qquad b \in U
}
{
\widehat{a \to b} \in U
}
&
$Set(\widehat{a \to b}) = Set(a) \to Set(b)$
\end{tabular}
\end{center}
~

\pause

\item Co dla $(\Pi x \in A) B(x)$ ?
\pause

\begin{center}
\begin{tabular}{lr}
\inference{
a \in U \qquad b \in Set(a) \to U
}
{
\widehat{\Pi(a,b)} \in U
}
\end{tabular}
\end{center}
\pause

\[
 Set(\widehat{\Pi(a,b)}) = (\Pi x \in Set(a))\; Set(b(x))
\]


\end{itemize}

\end{frame}

%%%%%%%%%%%%%%%%%%%%%%%%%

\begin{frame}
\frametitle{Uniwersum ala Tarski}

\begin{itemize}
 \item Bez uniwersów nie umiemy udowodnić że konstruktory tworzą różne wartości!
 \item celujemy w sąd $\neg [0 =_{Nat} succ(n)]\;[n \in Nat]$
 \item niech $n \in Nat$, $e \in [0 =_{Nat} succ(n)]$

\begin{eqnarray*}
 IsZ(m) &\equiv& Set(natrec(m, \widehat{N_1}, (x,y)\widehat{N_0})) \\
 IsZ(0) &\equiv& N_1 \\
 IsZ(succ(n)) &\equiv& N_0
\end{eqnarray*}

\begin{center}
\begin{tabular}{c}
\inference{
e \in [0 =_{Nat} succ(n)] \qquad 0_1 \in IsZ(0)
}
{
subst(e,0_1) \in IsZ(succ(n))
}
\end{tabular}
\end{center}

\item $\lambda n. \lambda e.\; subst(e, 0_1) \in (\Pi n \in Nat) \neg [0 =_{Nat} succ(n)]$
\end{itemize}

\end{frame}

%%%%%%%%%%%%%%%%%%%%%%%%%

\begin{frame}
\frametitle{Uniwersum ala Russel}

\begin{center}
\begin{tabular}{lr}
\inference{
}
{
U\;set
}
&
\inference{
}
{
Nat \in U
}
\end{tabular}
\end{center}

\begin{center}
\begin{tabular}{lcr}
\inference{
}
{
N_0\in U
}
&
\inference{
}
{
N_1\in U
}
&
\inference{
}
{
N_2\in U
}
\end{tabular}
\end{center}

\[ \cdots \]

\begin{center}
\begin{tabular}{lr}
\inference{
A \in U \qquad B(x) \in U\;[x \in A]
}
{
(\Pi x \in A) B(x) \in U
}
&
\inference{
A \in U \qquad B(x) \in U\;[x \in A]
}
{
(\Sigma x \in A) B(x) \in U
}
\end{tabular}
\end{center}

\begin{center}
\begin{tabular}{lcr}
\inference{
a \in U
}
{
a\;set
}
&
\inference{
a = b \in U
}
{
a = b
}
\end{tabular}
\end{center}


\end{frame}

%%%%%%%%%%%%%%%%%%%%%%%%%


\begin{frame}
\frametitle{Uniwersum ala Russel}

\begin{itemize}
 \item wygodne
 \item różnica w sensie: posługiwanie się nazwami typów a posługiwanie się typami

\begin{itemize}
 \item pokazanie $\neg [ \widehat{N} =_U \widehat{Bool} ]$ to jak pokazanie $\neg [ 0 =_{Nat} 1 ]$
 \item pokazanie $\neg [ N =_U Bool ]$ nie jest takie proste
 \[ \forall b \in Bool.\; [ b =_{Bool} true ] \vee [ b =_{Bool} false ] \]
\end{itemize}

\end{itemize}
\end{frame}

%%%%%%%%%%%%%%%%%%%%%%%%%

%%%%%%%%%%%%%%%%%%%%%%%%%

\begin{frame}
\frametitle{Uniwersum ala Russel}

\[
 U_0 = U
\]

\begin{center}
\begin{tabular}{lr}
\inference{
}
{
U_n\;set
}
\end{tabular}
\end{center}

\begin{center}
\begin{tabular}{lr}
\inference{
A \in U_n \qquad B(x) \in U_n\;[x \in A]
}
{
(\Pi x \in A) B(x) \in U_n
}
&
\inference{
A \in U_n \qquad B(x) \in U_n\;[x \in A]
}
{
(\Sigma x \in A) B(x) \in U_n
}
\end{tabular}
\end{center}


\begin{center}
\begin{tabular}{lr}
\inference{
}
{
U_n \in U_{n+1}
}
\end{tabular}
\end{center}

\end{frame}

%%%%%%%%%%%%%%%%%%%%%%%%%


%%%%%%%%%%%%%%%%%%%%%%%%%
%%%%%%%%%%%%%%%%%%%%%%%%%
%%%%%%%%%%%%%%%%%%%%%%%%%

\begin{frame}
\frametitle{W-typy}

\begin{center}
\begin{tabular}{c}
\inference{
A\;set \qquad B(x)\;set\;[x \in A]
}
{
(Wx \in A) B(x)\;set 
}
\end{tabular}
\end{center}

\begin{center}
\begin{tabular}{c}
\inference{
a \in A \qquad b(x) \in (W x \in A)\;B(x)\;[x \in B(a)]
}
{
sup(a,b) \in (W x \in A) B(x)
}
\end{tabular}
\end{center}


\begin{itemize}
 \item dobrze ufundowane drzewa
 \item kodowanie typów indukcyjnych
\end{itemize}

\end{frame}


%%%%%%%%%%%%%%%%%%%%%%%%%

\begin{frame}
\frametitle{W-typy}

\begin{eqnarray*}
B(x) &\equiv& Set(case_2(x, \hat{N_0}, \hat{N_1})) \\
Nat     &=& (Wx \in N_2) B(x) \\
zero    &=& sup(0_2, case_0) \\
succ(n) &=& sup(1_2, (x) n)
\end{eqnarray*}

\end{frame}

%%%%%%%%%%%%%%%%%%%%%%%%%

\begin{frame}
\frametitle{W-typy}

\begin{center}
\begin{tabular}{c}
\inference{
a \in (Wx \in A) B(x) \\
C(v)\;set\;[v \in (Wx \in A) B(x)] \\
b(y, z, u) \in C(sup(y,z))\;\\[\\
 ~\qquad y \in A, \\
 ~\qquad z(x) \in (Wx \in A)B(x)\;[x \in B(y)] \\
 ~\qquad u(x) \in C(z(x))\;[x\in B(y)] \\
]
}
{
wrec(a,b) \in C(a)
}
\end{tabular}
\end{center}

\end{frame}

%%%%%%%%%%%%%%%%%%%%%%%%%
%%%%%%%%%%%%%%%%%%%%%%%%%

\begin{frame}
\frametitle{W-typy}

\begin{center}
\begin{tabular}{c}
\inference{
d \in A\qquad e(x) \in (Wx \in A) B(x)\;[x \in B(d)] \\
C(v)\;set\;[v \in (Wx \in A) B(x)] \\
b(y, z, u) \in C(sup(y,z))\;\\[\\
 ~\qquad y \in A, \\
 ~\qquad z(x) \in (Wx \in A)B(x)\;[x \in B(y)] \\
 ~\qquad u(x) \in C(z(x))\;[x\in B(y)] \\
]
}
{
wrec(sup(d,e),b) = b(d,e,(x)wrec(e(x), b)) \in C(sup(d,e))
}
\end{tabular}
\end{center}

\end{frame}

%%%%%%%%%%%%%%%%%%%%%%%%%

\begin{frame}
\frametitle{W-typy}

\begin{eqnarray*}
B(x) &\equiv& Set(case_2(x, \hat{N_0}, \hat{N_1})) \\
Nat           &=& (Wx \in N_2) B(x) \\
zero          &=& sup(0_2, case_0) \\
succ(n)       &=& sup(1_2, (x) n) \\
natrec(a,b,c) &=& wrec(a, (y,z,u)case_2(y, b, c(z(tt), u(tt))))
\end{eqnarray*}

\begin{itemize}
 \item równość ekstensjonalna lub aksjomat ekstensjonalności funkcji
\end{itemize}


\end{frame}

%%%%%%%%%%%%%%%%%%%%%%%%%


\end{document}
